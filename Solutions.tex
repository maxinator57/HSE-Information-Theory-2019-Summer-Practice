\documentclass{article}
\usepackage[left=10mm, top=10mm, right=20mm, bottom=10mm, nohead, nofoot]{geometry}
\usepackage{amsmath,amsthm,amssymb}
\usepackage[utf8]{inputenc}
\usepackage[russian]{babel}
\usepackage{graphicx}
\usepackage{listings}
\DeclareMathOperator{\grad}{grad}
\DeclareMathOperator{\E}{\mathsf{E}}
\DeclareMathOperator{\D}{\mathsf{D}}
\DeclareMathOperator{\cov}{\mathsf{cov}}
\renewcommand{\ge}{\geqslant}
\renewcommand{\le}{\leqslant}
\newcommand{\same}{\Leftrightarrow}
\renewcommand{\d}{\ {\rm{d}}}
\newcommand{\osmall}{\bar o}
\newcommand{\eps}{\varepsilon}
\newcommand{\ds}{\displaystyle}
\newcommand{\Exp}{\mathsf{Exp}}
\begin{document}
\author{Урманов Максим Тимурович, ПМИ ФКН, группа 171-2}
\title{Решения задач из главы 2 книги \\Thomas M. Cover, Joy A. Thomas \\Elements of Information Theory, Second Edition}
\date{Июль 2019 г.}
\maketitle
\parindent=0cm
\begin{enumerate}
\item[\bfseries 3'.] Найти максимальное возможное значение энтропии для дискретного распределения с $n$ значениями.
\\\\$\vartriangleright$  Пусть $p_i$ --- вероятность $i$-го значения. Тогда энтропия записывается в виде
$$H(p_1, \ldots, p_n) = -\sum_{i = 1}^n p_i \log p_i, \text{ где } \sum_{i = 1}^n p_i = 1.$$
Покажем, что максимум энтропии равен $\log n$ и достигается тогда и только тогда, когда все $p_i$ равны $1/n$.
\\Рассмотрим произвольный набор неотрицательных значений $p_i,$ где $\ds \sum_{i = 1}^n p_i = 1$. Пусть не все $p_i$ равны $1/n$. 
\\Попробуем заменить часть значений $p_i$ так, чтобы сумма всех $p_i$ осталась прежней, а энтропия увеличилась. Так как не все $p_i$ равны $1/n$, найдутся такие $m$ и $k$, что $p_m < 1/n < p_k$. Заменим $p_m$ и $p_k$ на $1/n$ и $s - 1/n$ соответственно и покажем, что после такой замены энтропия увеличится. Для краткости обозначим $s = p_m + p_k$ и рассмотрим функцию $$f(x) = -x\log x - (s - x)\log (s - x),$$ определённую на отрезке $[0, s]$. Очевидно, $f(p_m) = f(p_k) = p_m \log p_m + p_k \log p_k$ и нужно лишь доказать, что $f(1/n) > f(p_m)$. Для этого найдём производную функции $f$:
$$f'(x) = -x \cdot \frac{1}{x} - 1 \cdot \log x - (s - x) \cdot \frac{-1}{s - x} - (-1) \cdot \log (s - x) = \log(s - x) - \log x.$$
Отсюда видно, что $f$ возрастает при $\ds 0 < x < \frac{s}{2}$ и убывает при $\ds \frac{s}{2} < x < s$. Так как $\ds p_m < \frac{s}{2} < p_k$, то $f$ возрастает от $p_m$ до $\ds \frac{s}{2}$ и убывает от $\ds \frac{s}{2}$ до $p_k$. Это значит, что на всём интервале $(p_m, p_k)$ выполнено $f(x) > f(p_m) = f(p_k)$. Осталось заметить, что $\ds p_m < \frac{1}{n} < p_k$, откуда $f(1/n) > f(p_m) = f(p_k)$, что и требовалось доказать.
\\\\Таким образом, в результате замены $p_m,\ p_k \to 1/n,\ s - 1/n$ энтропия увеличилась. Сделав так не более, чем $(n - 1)$ замен, мы получим распределение, где все $p_i$ равны $1/n$, причем при каждой замене энтропия строго увеличивалась. Значит, максимальное значение энтропии равно $$\sum_{i = 1}^n -\frac{1}{n}\log \frac{1}{n} = - \log \frac{1}{n} = \log n$$
и достигается лишь в случае, когда все $p_i$ равны $1/n$. $\square$
\\
\item[\bfseries 29.] \textit{Неравенства.} Доказать неравенства:
\begin{enumerate}
\item[\bfseries (a)] $H(X, Y | Z) \ge H(X | Z).$
\\\\$\vartriangleright$ По цепному правилу для условной совместной энтропии, имеем 
$$H(X, Y | Z) = H(X | Z) + H(Y|X, Z) \ge H(X|Z),$$
так как энтропия (в том числе условная) всегда неотрицательна. $\square$
\\
\item[\bfseries (b)] $I(X, Y; Z) \ge I(X; Z).$
\\\\$\vartriangleright$ Применяя цепное правило для взаимной информации, имеем
$$I(X, Y; Z) = I(Y; Z | X) + I(X; Z) \ge I(X; Z).$$
Последний переход верен в силу того, что взаимная информация (в том числе условная) всегда неотрицательна. $\square$
\\
\item[\bfseries (c)] $H(X, Y, Z) - H(X, Y) \le H(X, Z) - H(X).$
\\\\$\vartriangleright$ Выведем это неравенство из предыдущего пункта. 
Для этого перепишем левую и правую части неравенства {\bfseries (b)}, используя равенство $(2.41)$ из книги <<Elements of Information Theory, 2nd edition>>:
$$I(X, Y; Z) = H(X, Y) + H(Z) - H(X, Y, Z),$$
$$I(X; Z) = H(X) + H(Z) - H(X, Z).$$
Тогда всё неравенство {\bfseries (b)} можно переписать в виде
$$H(X, Y) + H(Z) - H(X, Y, Z) \ge H(X) + H(Z) - H(X, Z) \Longleftrightarrow H(X, Y, Z) - H(X, Y) \le H(X, Z) - H(X),$$
что и требовалось доказать. $\square$
\\
\item[\bfseries (d)] $I(X; Z | Y) \ge I(Z; Y | X) - I(Z; Y) + I(X; Z).$
\\\\$\vartriangleright$ Покажем, что это неравенство на самом деле всегда обращается в равенство. Перенесём $-I(Z; Y)$ из правой части в левую и заметим, что
$$I(X; Z | Y) + I(Z; Y) = I(X; Z | Y) + I(Y; Z) = |\text{по цепному правилу для информации}| = I(X, Y; Z),$$
$$I(Z; Y | X) + I(X; Z) = I(Y; Z | X) + I(X; Z) = |\text{по цепному правилу для информации}| = I(Y, X; Z).$$
Осталось заметить, что, очевидно, $\ds I(X, Y; Z) = I(Y, X; Z).\ \square$
\\ \item[\bfseries 29.] \textit{Неравенства.} Доказать неравенства:
\begin{enumerate}
\item[\bfseries (a)] $H(X, Y | Z) \ge H(X | Z).$
\\\\$\vartriangleright$ По цепному правилу для условной совместной энтропии, имеем 
$$H(X, Y | Z) = H(X | Z) + H(Y|X, Z) \ge H(X|Z),$$
так как энтропия (в том числе условная) всегда неотрицательна. $\square$
\\
\item[\bfseries (b)] $I(X, Y; Z) \ge I(X; Z).$
\\\\$\vartriangleright$ Применяя цепное правило для взаимной информации, имеем
$$I(X, Y; Z) = I(Y; Z | X) + I(X; Z) \ge I(X; Z).$$
Последний переход верен в силу того, что взаимная информация (в том числе условная) всегда неотрицательна. $\square$
\\
\item[\bfseries (c)] $H(X, Y, Z) - H(X, Y) \le H(X, Z) - H(X).$
\\\\$\vartriangleright$ Выведем это неравенство из предыдущего пункта. 
Для этого перепишем левую и правую части неравенства {\bfseries (b)}, используя равенство $(2.41)$ из книги <<Elements of Information Theory, 2nd edition>>:
$$I(X, Y; Z) = H(X, Y) + H(Z) - H(X, Y, Z),$$
$$I(X; Z) = H(X) + H(Z) - H(X, Z).$$
Тогда всё неравенство {\bfseries (b)} можно переписать в виде
$$H(X, Y) + H(Z) - H(X, Y, Z) \ge H(X) + H(Z) - H(X, Z) \Longleftrightarrow H(X, Y, Z) - H(X, Y) \le H(X, Z) - H(X),$$
что и требовалось доказать. $\square$
\\
\item[\bfseries (d)] $I(X; Z | Y) \ge I(Z; Y | X) - I(Z; Y) + I(X; Z).$
\\\\$\vartriangleright$ Покажем, что это неравенство на самом деле всегда обращается в равенство. Перенесём $-I(Z; Y)$ из правой части в левую и заметим, что
$$I(X; Z | Y) + I(Z; Y) = I(X; Z | Y) + I(Y; Z) = |\text{по цепному правилу для информации}| = I(X, Y; Z),$$
$$I(Z; Y | X) + I(X; Z) = I(Y; Z | X) + I(X; Z) = |\text{по цепному правилу для информации}| = I(Y, X; Z).$$
Осталось заметить, что, очевидно, $\ds I(X, Y; Z) = I(Y, X; Z).\ \square$
\\
\end{enumerate}
\end{enumerate}
\item[\bfseries 23.] \textit{Условная взаимная информация.} Рассмотрим последовательность из $n$ бинарных случайных величин $X_1. \ldots X_n$. Вероятность каждой последовательности с чётным числом единиц равна $1/2^{n - 1}$, а вероятность каждой последовательности с нечётным числом единиц равна $0$. Найти взаимные информации $$I(X_1; X_2),\ I(X_2; X_3|X_1), \ldots,\ I(X_{n - 1}; X_n | X_1, \ldots, X_{n - 2}).$$
$\vartriangleright$  Покажем, что случайные величины $X_1, \ldots X_{n - 1}$ независимы в совокупности. Для этого достаточно доказать, что для любого $k < n$ случайная величина $X_k$ независима со случайным вектором $(X_1, \ldots, X_{k - 1})$. 
\\В свою очередь, это эквивалентно тому, что при всех значениях $\bar a = (a_1, \ldots, a_{k - 1}) \in \{0, 1\}^{k - 1}$ вектора $(X_1, \ldots, X_{k - 1})$ условные вероятности $\Prob[X_k = 0\ |\ (X_1, \ldots, X_{k - 1}) = \bar a]$ и $\Prob[X_k = 1\ |\ (X_1, \ldots, X_{k - 1}) = \bar a]$ равны $1/2$. 
\\\\Будем доказывать последнее утверждение индукцией по $k$. База для $k = 1$ будет доказываться так же, как и переход, с учётом того, что $1/2^0 = 1$. Поэтому сразу докажем переход. 
\\\\Пусть $X_1, \ldots X_{k - 1}$ независимы в совокупности. Предположим, что $X_{k + 1}$ не независима с ними, то есть для некоторого бинарного вектора $\bar a = (a_1, \ldots, a_{k - 1})\ \Prob[X_k = a_k\ |\ (X_1, \ldots X_{k - 1}) = \bar a] = p > 1/2$. Тогда рассмотрим такое значение $a_{k + 1} \in \{0, 1\}$ случайной величины $X_{k + 1}$, что $\Prob[X_{k + 1} = a_{k + 1}\ |\ X_1 = a_1, \ldots,\ X_k = a_k] \ge 1/2.$ Далее рассмотрим аналогичное значение $a_{k + 2}$ для $X_{k + 2}$ (то есть такое, что $\Prob[X_{k + 2} = a_{k + 2}\ |\ X_1 = a_1, \ldots,\ X_k = a_k] \ge 1/2$), потом аналогичное значение $a_{k + 3}$ для $X_{k + 3}$ и т. д. до $X_{n - 1}$. Очевидно, такие значения $a_{k + i}$ всегда найдутся, так как сумма соответствующих условных вероятностей для $a_{k + i} = 0$ и $a_{k + i} = 1$ равна 1. Тогда имеем
$$\Prob[(X_1, \ldots, X_{n - 1}) = (a_1, \ldots, a_{n - 1})] =$$
$$= \Prob[(X_1, \ldots, X_{k - 1}) = (a_1, \ldots, a_{k - 1})] \cdot \prod_{i = k}^{n - 1} \Prob[X_i = a_i\ |\ (X_1, \ldots, X_{i - 1}) = (a_1, \ldots, a_{i - 1})] =$$
\newpage
$$= |\text{из независимости $X_1, \ldots, X_{k - 1}$ в совокупности}| =$$
$$= \left(\frac{1}{2}\right)^{k - 1} \cdot\ p\ \cdot \prod_{i = k + 1}^{n - 1} \Prob[X_i = a_i\ |\ (X_1, \ldots, X_{i - 1}) = (a_1, \ldots, a_{i - 1})] \ge p \cdot \left(\frac{1}{2}\right)^{n - 2}.$$
Заметим, что, поскольку вероятность вектора из нечётного числа единиц равна $0$ (а из нечётного -- не равна 0), то условные вероятности $\Prob[X_n = 0\ |\ (X_1, \ldots, X_{n - 1}) = (a_1, \ldots, a_{n - 1})]$ и $\Prob[X_n = 1\ |\ (X_1, \ldots, X_{n - 1}) =\\= (a_1, \ldots, a_{n - 1})]$ равны $1$ и $0$ в каком-то порядке. Значит, выбирая $a_n$ таким образом, чтобы соответствующая вероятность для $a_n$ была равна $1$, получаем
$$\Prob[(X_1, \ldots, X_{n}) = (a_1, \ldots, a_{n - 1}, a_n)] =$$
$$= \Prob[(X_1, \ldots, X_{n - 1}) = (a_1, \ldots, a_{n - 1})] \cdot \Prob[X_n = a_n\ |\ (X_1, \ldots, X_{n - 1}) = (a_1, \ldots, a_{n - 1})] =$$
$$= \Prob[(X_1, \ldots, X_{n - 1}) = (a_1, \ldots, a_{n - 1})] \cdot 1 \ge p \cdot \left(\frac{1}{2}\right)^{n - 2} > \left(\frac{1}{2}\right)^{n - 1},$$
 что противоречит условию задачи. Тогда предположение неверно и переход доказан.
\\\\
Заметим, что мы на самом деле доказали, что любые $n - 1$ случайных величин из $X_1, \ldots, X_n$ независимы в совокупности. Это следует из того, что все эти случайные величины равноправны.
\\\\Для решения задачи осталось заметить, что все условные информации вида $I(X_k; X_{k + 1}|X_1, \ldots, X_{k - 1})$ при $k < n - 1$ равны $0$, так как случайные величины $X_1, \ldots, X_{k + 1}$ при $k < n - 1$ независимы в совокупности. 
\\В то же время, имеем 
$$I(X_{n - 1}; X_{n}|X_1, \ldots,  X_{n - 2}) = H(X_{n - 1} |X_1, \ldots, X_{n - 2}) - H(X_{n - 1} |X_1, \ldots, X_{n - 2}, X_{n}).$$
В получившейся разности уменьшаемое равно 1, так как условие никак не влияет на распределение и по сути у нас будет просто энтропия честной монетки (из доказательства независимости величин $X_1, \ldots, X_{n - 1}$ в совокупности и самой независимости в совокупности следует, что для любого $k\ \Prob[X_k = 0] = \Prob[X_k = 1] = 1/2).$ Вычитаемое же очевидно равно 0, так как $X_{n - 1}$ явно определяется через остальные $X_i,\ i \in \{1, \ldots, n - 2\} \cup \{ n\}.$
\\\\Значит, все взаимные информации, кроме $I(X_{n - 1}; X_{n}|X_1, \ldots, X_{n - 2}),$ равны $0$, а последняя равна $1$. $\square$
\\
\item[\bfseries 25.] \textit{Диаграммы Венна} (определение взаимной информации для трёх случайных величин).
\\Взаимная информация для трёх случайных величин определяется как
$$I(X; Y; Z) \overset{\rm def}{=} I(X; Y) - I(X; Y | Z).$$
\begin{enumerate}
\item[\bfseries (a)] Привести пример случайных величин $X,\ Y,\ Z$, таких что $I(X; Y; Z) < 0$.
\\\\$\vartriangleright$ Это в точности задача 6b. Подойдёт пример $Z = X + Y$, где $X$ и $Y$ --- независимые <<честные монетки>> $Bernoulli(1/2)$. $\square$
\\
\item[\bfseries (b)] Доказать равенство $I(X; Y; Z) = H(X, Y, Z) - H(X) - H(Y) - H(Z) + I(X; Y) + I(Y; Z) + I(Z; X).$
\\\\$\vartriangleright$ По определению взаимной информации для трёх случайных величин имеем
\begin{equation} I(X; Y; Z) = I(X; Y) - I(X; Y | Z) = I(X; Y) - H(X|Z) + H(X|Y, Z). \end{equation}
Используя цепное правило для энтропии, имеем
$$H(X, Y, Z) = H(X|Y, Z) + H(Y | Z) + H(Z),$$ откуда
\begin{equation} H(X | Y, Z) = H(X, Y, Z) - H(Y|Z) - H(Z).\end{equation}
Подставляя правую часть (2) вместо $ H(X | Y, Z)$ в (1), получаем
$$I(X; Y; Z) = I(X; Y) - H(X|Z) + H(X, Y, Z) - H(Y|Z) - H(Z) =$$
\newpage
\begin{equation} = H(X, Y, Z) - H(Z) + I(X; Y) - H(X|Z) - H(Y|Z). \end{equation}
Наконец, используя известное равенство для взаимной информации ((2.43) в книге)
$$I(A; B) = H(A) - H(A|B) \Longleftrightarrow -H(A|B) = I(A; B) - H(A)$$  и применяя его к $(A, B) = (X, Z)$ и $(A, B) = (Y, Z)$, а после подставляя в (3), получаем
$$I(X; Y; Z) = H(X, Y, Z) - H(Z) + I(X; Y) + \Big(I(X; Z) - H(X)\Big) + \Big(I(Y; Z) - H(Y)\Big) =$$
$$= H(X, Y, Z) - H(X) - H(Y) - H(Z) + I(X; Y) + I(Y; Z) + I(Z; X),$$
что и требовалось. С точки зрения диаграмм Венна это равенство можно интерпретировать как формулу включений-исключений, так как $I(X; Y)$ с точки зрения диаграмм означает <<пересечение>> $H(X)$ и $H(Y)$, а $I(X; Y; Z)$ по логике должно означать пересечение $H(X),\ H(Y)$ и $H(Z)$. $\square$
\\
\item[\bfseries (c)] Доказать равенство $I(X; Y; Z) = H(X, Y, Z) - H(X, Y) - H(Y, Z) - H(Z, X) + H(X) + H(Y) + H(Z).$
\\\\$\vartriangleright$ Выведем его из равенства {\bfseries (b)}. Для этого вспомним известное равенство ((2.45) в книге)
$$I(A; B) = H(A) + H(B) - H(A, B).$$
Применяя его поочерёдно к $(A, B) = (X, Y),\ (Y, Z)$ и $(Z, X)$ и подставляя в правую часть равенства {\bfseries (b)} вместо $I(X; Y),\ I(Y; Z)$ и $I(Z; X)$ соответственно, получаем
$$I(X; Y; Z) = H(X, Y, Z) - H(X) - H(Y) - H(Z) + \Big(H(X) + H(Y) - H(X, Y)\Big) +$$
$$+\ \Big(H(Y) + H(Z) - H(Y, Z)\Big) + \Big(H(Z) + H(X) - H(Z, X)\Big) = $$
$$= H(X, Y, Z) - H(X, Y) - H(Y, Z) - H(Z, X) + H(X) + H(Y) + H(Z),$$
что и требовалось. $\square$
\\
\end{enumerate}
\item[\bfseries 32.] \textit{Фано.} Случайные величины $X$ и $Y$ имеют следующее совместное распределение:
\begin{center}
\begin{tabular}{ c c | c c c}
& & & $Y$ & \\
& & $a$ & $b$ & $c$\\ \hline
& 1 & $1/6$ & $1/12$ & $1/12$ \\
$X$ & 2& $1/12$ & $1/6$ & $1/12$\\
& 3 & $1/12$ & $1/12$ & $1/6$\\
\end{tabular}
\end{center}
\begin{enumerate}
\item[\bfseries (a)] Найти оценку $\hat X(Y)$ для $X$ с минимальной вероятностью ошибки.
\\\\$\vartriangleright$ Очевидно, можно искать оценку $\hat X(Y)$ отдельно для каждого значения $Y$. Для примера, рассмотрим $Y = a$ (в силу симметричности совместного распределения оптимальные вероятности ошибки для $Y = b$ и $Y = c$ будут такими же). Любая оценка $\hat X(a)$ имеет вид 
$$\hat X(a) = 
\begin{cases}
1\ \text{с вероятностью } p_1\\
2\ \text{с вероятностью } p_2\\
3\ \text{с вероятностью } p_3 = 1 - p_1 - p_2\\
\end{cases}$$
Найдём вероятность ошибки такой оценки. Но проще найти вероятность не ошибиться. Имеем 
$$\Prob[X = \hat X(a)\ |\ Y = a] = \Prob[X = 1, \hat X(a) = 1\ |\ Y = a] + \Prob[X = 2, \hat X(a) = 2\ |\ Y = a] + \Prob[X = 3, \hat X(a) = 3\ |\ Y = a] =$$
$$= p_1 \cdot \frac{1}{2} + p_2 \cdot \frac{1}{4} + (1 - p_1 - p_2) \cdot \frac{1}{4} = \frac{1}{4} + \frac{1}{4}p_1.$$
Таким образом, вероятность не ошибиться максимальна при $p_1 = 1$ и равна $1/2$, а значит, любая оценка $\hat X(a)$ ошибается с вероятностью не меньше, чем $1/2$. То же самое верно и для оценок $\hat X(b)$ и $\hat X(c)$, поэтому вероятность ошибки оценки $\hat X(Y)$ не меньше, чем $1/2.$ Из рассуждения выше ясно, что вероятность ошибки $1/2$ достигается тогда и только тогда, когда $\hat X(a) = 1,\ \hat X(b) = 2$ и $\hat X(c) = 3$ с вероятностью 1. $\square$
\\\\
\item[\bfseries (b)] Записать неравенство Фано для пункта {\bfseries (a)} и сравнить результаты.
\\\\$\vartriangleright$ Воспользуемся ослабленным неравенством Фано в форме 
$$P_e \ge \frac{H(X|Y) - 1}{\log |\chi|},$$
где $\chi$ --- множество значений случайной величины $X$. Имеем $|\chi| = 3$ и $$H(X|Y) = H(X, Y) - H(Y) = -\frac{3}{6} \cdot \log\frac{1}{6} - \frac{6}{12} \cdot \log\frac{1}{12} + 3 \cdot \frac{1}{3}\log\frac{1}{3} =$$
$$= \frac{1}{2}(1 + \log 3) + \frac{1}{2}(2 + \log 3) - \log 3 = \frac{3}{2} + \log 3 - \log 3 = \frac{3}{2}.$$
Подставляя полученные значения в неравенство Фано, получаем следующую оценку на вероятность ошибки:
$$P_e \ge \frac{3/2 - 1}{\log 3} = \frac{1}{2\log 3},$$
что заметно меньше, чем настоящая минимальная возможная вероятность ошибки. $\square$
\\
\end{enumerate} 
\item[\bfseries 37.] \textit{Относительная энтропия.} Пусть $X,\ Y,\ Z$ --- случайные величины с совместной функцией распределения $p(x, y, z).$ Относительная энтропия совместной функции вероятности к произведению маргинальных функций вероятности для этой тройки по определению равна
$$D\Big(P(x, y, z)||p(x)p(y)p(z)\Big) = \E \left[\log \frac{p(x, y, z)}{p(x)p(y)p(z)}\right].$$
Выразить это значение через энтропии. Когда оно равно нулю?
\\\\$\vartriangleright$ Имеем
$$D\Big(P(x, y, z)||p(x)p(y)p(z)\Big) = \E \left[\log \frac{p(x, y, z)}{p(x)p(y)p(z)}\right] = \sum_{x, y, z} p(x, y, z)\log \frac{p(x, y, z)}{p(x)p(y)p(z)} =$$
$$= \sum_{x, y, z} p(x, y, z)\log p(x, y, z) - \sum_{x, y, z} p(x, y, z)\log p(x) - \sum_{x, y, z} p(x, y, z)\log p(y) -\sum_{x, y, z} p(x, y, z)\log p(z) = $$
$$=-H(X, Y, Z) - \sum_{x} p(x, y, z)\log p(x) - \sum_{y} p(y)\log p(y) - \sum_{z} p(z)\log p(z) = -H(X, Y, Z) + H(X) + H(Y) + H(Z).$$
Поймём, когда это выражение равно $0$.
По цепному правилу для энтропии, имеем
$$H(X, Y, Z) = H(X) + H(Y|X) + H(Z|Y, X),$$ откуда
$$D\Big(P(x, y, z)||p(x)p(y)p(z)\Big) = H(X) + H(Y) + H(Z) - H(X) - H(Y|X) - H(Z|Y, X) = $$
$$=\Big(H(Y) - H(Y|X)\Big) + \Big(H(Z) - H(Z|Y, X)\Big).$$
Так как дополнительное условие может только уменьшить энтропию (неравенство (2.95) из книги), то оба слагаемых в последней сумме неотрицательны. Первое из них обращается в $0$ тогда и только тогда, когда $X$ и $Y$ независимы, а второе --- когда $Z$ и случайный вектор $(X, Y)$ независимы. Из этих двух условий и определения независимости в совокупности легко следует независимость $X,\ Y$ и $Z$ в совокупности. Обратное следствие очевидно. Таким образом, $D\Big(P(x, y, z)||p(x)p(y)p(z)\Big) = 0$ тогда и только тогда, когда $X,\ Y$ и $Z$ независимы в совокупности. $\square$
\\
\item[\bfseries 43.] \textit{Взаимная информация орлов и решек.}
\begin{enumerate}
\item[\bfseries (a)] Рассмотрим подбрасывание честной монетки. Найти взаимную информацию верхней и нижней сторон монетки.
\\\\$\vartriangleright$ Всего есть два исхода с ненулевой вероятностью: (орёл сверху, решка снизу) и (решка сверху, орёл снизу). Так как вероятность каждого из них равна $1/2$, то взаимная информация равна
$$I = 2 \cdot \frac{1}{2} \cdot \log \left(\frac{1/2}{(1/2)^2}\right) = \log 2 = 1.$$
$\square$
\\
\item[\bfseries (b)] Бросается честный шестигранный кубик. Найти взаимную информацию верхней и передней граней.
\\\\$\vartriangleright$ Здесь такая же логика, как для монетки. Нужно лишь заметить, что любая возможная пара из верхней и передней грани задаётся их общим ребром и одной из двух возможных ориентаций (какая грань верхняя, а какая --- передняя). Все такие пары равновероятны, а их число равно удвоенному числу рёбер, то есть $12 \cdot 2 = 24$. При этом вероятность того, что фиксированная грань оказалась верхней (аналогично, передней), равна $1/6$, поэтому взаимная информация равна
$$I = 24 \cdot \frac{1}{24} \cdot \log\left(\frac{1/24}{(1/6)^2}\right) = \log \frac{36}{24} = \log \frac{3}{2} = \log 3 - 1.$$
$\square$
\\
\end{enumerate}
\item[\bfseries 44.] \textit{Чистый рандом.} Пусть $X$ --- трёхсторонняя монетка с распределением
$$X = \begin{cases}A,\ p_A\\ B,\ p_B\\ C,\ p_C, \end{cases}$$ где $p_A,\ p_B$ и $p_C$ неизвестны.
\begin{enumerate}
\item[\bfseries (a)] Используя два независимых подбрасывания, сгенерировать распределение $Bernoulli(1/2)$.
\\\\$\vartriangleright$ Заметим, что для каждого исхода из двух подбрасываний, когда выпадают разные результаты, перестановкой результатов получается исход с такой же вероятностью. Например, для исхода $AB$ получится исход $BA$ и эти исходы оба имеют вероятность $p_Ap_B$. Тогда можно взять все исходы, когда выпадают разные результаты,  разбить их на две равновероятные группы и считать, что при выпадении исхода из первой группы <<выпадает орёл>>, а при выпадении исхода из второй группы --- решка. Например, подойдёт разбиение на $\{AB,\ BC,\ CA\}$ и $\{BA,\ CB,\ AC\}.$ В то же время, с исходами $AA,\ BB$ и $CC$ мы ничего не можем сделать, поэтому их мы будем просто игнорировать --- считать, что если выпал один из них, то генерация не удалась. $\square$
\\
\item[\bfseries (b)] Какое максимальное ожидаемое число честных бит может быть так сгенерировано?
\\\\$\vartriangleright$ Честный бит будет сгенерирован если и только если два броска покажут разные результаты. Вероятность этого равна $1 - (p_A^2 + p_B^2 + p_C^2)$. Это и есть ожидаемое число сгенерированных честных бит. Так как $p_A + p_B + p_C = 1$, то по неравенству о средних имеем 
$$\sqrt{\frac{p_A^2 + p_B^2 + p_C^2}{3}} \ge \frac{p_A + p_B + p_C}{3} = \frac{1}{3},$$ откуда $$p_A^2 + p_B^2 + p_C^2 \ge 3 \cdot \left(\frac{1}{3}\right)^2 = \frac{1}{3} \Longleftrightarrow 1 - (p_A^2 + p_B^2 + p_C^2) \le 1 - \frac{1}{3} = \frac{2}{3}$$
и равенство достигается тогда и только тогда, когда $p_A = p_B = p_C = 1/3.\ \square$
\\\\
\end{enumerate}
\item[\bfseries 47.] \textit{Энтропия почти отсортированной колоды.} Дана отсортированная в возрастающем порядке колода из $n$ карт с номерами от $1$ до $n$. Из колоды равновероятно выбирается карта, вынимается, а потом вставляется в случайное место. Найти энтропию получающейся колоды.
\\\\$\vartriangleright$ Зафиксируем карту, которую вынули. Всего есть $n$ мест для вставки, поэтому может получиться $n$ разных колод, каждая с вероятностью $1/n$. В то же время, некоторые колоды, получающиеся в итоге при вынимании разных карт, могут совпасть. Поймём, когда это происходит. Ясно, что каждая колода --- это просто перестановка на $n$ элементах. Для произвольной нетождественной перестановки $f$ определим \textit{первое несоответствие} как $$m(f) = f(\min \{i: f(i) \ne i\}).$$
\\Иными словами, это образ минимального элемента, такого что этот образ не равен самому этому элементу. 
\newpage Легко понять, что для колоды, полученной после вынимания карты с номером $i$, первое несоответствие равно $i$, если карту вставили на позицию левее исходной, и $i + 1$, если вставили на позицию правее исходной. Отсюда очевидно следует, что одинаковые колоды (не считая тождественной перестановки) могли получиться только при вынимании соседних карт ($i$ и $i + 1$), причём карта $i$ должна была переместиться вправо, а карта $i + 1$ --- влево. Но заметим, что при перемещении карты $i$ вправо карта $i - 1$ по-прежнему останется левее, чем $i + 1$ (если карты $i - 1$ нет, работает такой же аргумент для перемещения карты $i + 1$ и карт $i$ и $i + 2$). Поэтому одинаковые перестановки могли получиться, только если карта $i$ поменялась местами с картой $i + 1$, то есть транспозиция $(i,\ i + 1)$ --- единственная общая перестановка. 
\\\\Теперь можно посчитать вероятности получения всех перестановок. Для тождественной перестановки она равна $n \cdot 1/n^2 = 1/n.$ Для транспозиций вида $(i,\ i + 1)$ вероятности равны $1/n^2 + 1/n^2 = 2/n^2$, а для всех остальных перестановок они равны $1/n^2$. Для подсчёта энтропии осталось найти число перестановок каждого типа. Тождественная перестановка единственна, а транспозиций вида $(i,\ i + 1)$ всего $n - 1$. Найдём число всех остальных возможных перестановок. При вытаскивании каждой карты получается $n$ перестановок, из них одна тождественная. Кроме того, для карт с номерами $1$ и $n$ среди них есть одна транспозиция, а для остальных карт --- две транспозиции. Итого получаем $2(n - 2) + (n - 2)(n - 3) = n^2 - 3n + 2$ перестановок.
\\\\Наконец, можно найти энтропию:
$$H = -\frac{1}{n}\log \frac{1}{n} - (n - 1) \cdot \frac{2}{n^2} \log \frac{2}{n^2} - (n^2 - 3n + 2) \cdot \frac{1}{n^2}\log \frac{1}{n^2} =$$
$$= \frac{1}{n}\log n + \frac{2(n - 1)(2\log n - 1)}{n^2} + \frac{(n^2 - 3n + 2) \cdot 2\log n}{n^2} = \left(2 - \frac{1}{n}\right)\log n - \frac{2n - 2}{n^2}.$$
$\square$
\\
\item[\bfseries 48.] \textit{Длина последовательности.} Сколько информации длина последовательности даёт о её элементах? Дан случайный процесс $\{X_n\}_{n = 1}^{\infty}$ с натуральным временем, для любого $n$ $X_n \sim Bernoulli(1/2)$. Как только появляется первая единица, процесс останавливается. Пусть $N$ --- номер шага, на котором процесс завершился, а $X^N$ --- случайный вектор, состоящий из значений $X_1, \ldots, X_N$. 
\begin{enumerate}
\item[\bfseries (a)] Найти $I(N; X^N)$.
\\\\$\vartriangleright$ Очевидно, каждому значению $N$ соответствует ровно одно значение $X^N$, а именно, вектор из $(N - 1)$ нулей и одной единицы. Вероятность того, что процесс продлится $N$ шагов, равна $1/2^N$ поэтому взаимная информация равна
$$I(N; X^N) = \sum_{N = 1}^\infty \frac{1}{2^N} \cdot \log \frac{1/2^N}{(1/2^N)^2} = \sum_{N = 1}^\infty \frac{1}{2^N} \log (2^N) = \sum_{N = 1}^\infty \frac{N}{2^N}.$$
Вспоминая, что $\ds \sum_{n = 1}^\infty n r^n = \frac{r}{(1 - r)^2},$ получаем $\ds I(N, X^N) = \frac{1/2}{(1 - 1/2)^2} = 2.$ $\square$
\\
\item[\bfseries (b)] Найти $H(X^N|N)$.
\\\\$\vartriangleright$ Так как $X^N$ есть функция от $N$, то такая условная энтропия равна нулю. $\square$
\\
\item[\bfseries (c)] Найти $H(X^N)$.
\\\\$\vartriangleright$ Так как $I(X; Y) = H(X) - H(X|Y)$ (равенство $(2.43)$ из книги), то имеем $$H(X^N) = I(X^N; N) + H(X^N|N) = I(N; X^N) + 0 = 2.$$
$\square$
\\\\
\end{enumerate}
Рассмотрим теперь другое время остановки. Будем останавливаться в момент $N = 6$ с вероятностью $1/3$ и в момент $N = 12$ с вероятностью $2/3$ независимо от значений членов последовательности $\{X_i\}_{i = 1}^{12}$.
\newpage
\begin{enumerate}
\item[\bfseries (d)] Найти $I(N; X^N)$.
\\\\$\vartriangleright$ Есть два возможных значения $N$, это $6$ и $12$. Для $N = 6$ есть $2^6$ равновероятных возможных значений $X^N$, для каждого такого значения $Y$ $p(N = 6,\ X^6 = Y) = 1/3 \cdot 1/2^6$. 
\\С другой стороны, $p(N = 6) = 1/3,\ p(X^N = Y) = 1/3 \cdot 1/2^6.$ Рассуждая аналогично для $N = 12$, получаем
$$I(N; X^N) = 2^6 \cdot \frac{1}{3} \cdot \frac{1}{2^6}\log \frac{1/3 \cdot 1/2^6}{1/3 \cdot  1/3 \cdot 1/2^6} + 2^{12} \cdot \frac{2}{3} \cdot \frac{1}{2^{12}}\log \frac{2/3 \cdot 1/2^{12}}{2/3 \cdot  2/3 \cdot 1/2^{12}} =$$ 
$$= \frac{1}{3}\log 3 + \frac{2}{3}(\log 3 - 1) = \log 3 - \frac{2}{3}.$$
$\square$
\\
\item[\bfseries (e)] Найти $H(X^N|N).$
\\\\$\vartriangleright$ Для любого бинарного вектора $Y$ длины $6$ $\Prob[X^N = Y | N = 6] = 1/2^6$, аналогично для любого вектора длины $12$ соответствующая условная вероятность равна $1/2^{12}$ Поэтому условная энтропия равна
$$H(X^N|N) = -2^6 \cdot \frac{1}{3} \cdot \frac{1}{2^6} \log \frac{1}{2^6} - 2^{12} \cdot \frac{2}{3} \cdot \frac{1}{2^{12}} \log \frac{1}{2^{12}} = 2 + 8 = 10.$$
$\square$
\\
\item[\bfseries (f)] Найти $H(X^N).$ 
\\\\$\vartriangleright$ Аналогично пункту {\bfseries (c)}, имеем
$$H(X^N) =  I(X^N; N) + H(X^N|N) = \log 3 - \frac{2}{3} + 10 = \log 3 + \frac{28}{3}.$$
$\square$
\end{enumerate}
\end{enumerate}
\end{document}